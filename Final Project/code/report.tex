% Options for packages loaded elsewhere
\PassOptionsToPackage{unicode}{hyperref}
\PassOptionsToPackage{hyphens}{url}
%
\documentclass[
]{article}
\title{Report}
\author{Huong Tran}
\date{11/24/2021}

\usepackage{amsmath,amssymb}
\usepackage{lmodern}
\usepackage{iftex}
\ifPDFTeX
  \usepackage[T1]{fontenc}
  \usepackage[utf8]{inputenc}
  \usepackage{textcomp} % provide euro and other symbols
\else % if luatex or xetex
  \usepackage{unicode-math}
  \defaultfontfeatures{Scale=MatchLowercase}
  \defaultfontfeatures[\rmfamily]{Ligatures=TeX,Scale=1}
\fi
% Use upquote if available, for straight quotes in verbatim environments
\IfFileExists{upquote.sty}{\usepackage{upquote}}{}
\IfFileExists{microtype.sty}{% use microtype if available
  \usepackage[]{microtype}
  \UseMicrotypeSet[protrusion]{basicmath} % disable protrusion for tt fonts
}{}
\makeatletter
\@ifundefined{KOMAClassName}{% if non-KOMA class
  \IfFileExists{parskip.sty}{%
    \usepackage{parskip}
  }{% else
    \setlength{\parindent}{0pt}
    \setlength{\parskip}{6pt plus 2pt minus 1pt}}
}{% if KOMA class
  \KOMAoptions{parskip=half}}
\makeatother
\usepackage{xcolor}
\IfFileExists{xurl.sty}{\usepackage{xurl}}{} % add URL line breaks if available
\IfFileExists{bookmark.sty}{\usepackage{bookmark}}{\usepackage{hyperref}}
\hypersetup{
  pdftitle={Report},
  pdfauthor={Huong Tran},
  hidelinks,
  pdfcreator={LaTeX via pandoc}}
\urlstyle{same} % disable monospaced font for URLs
\usepackage[margin=1in]{geometry}
\usepackage{graphicx}
\makeatletter
\def\maxwidth{\ifdim\Gin@nat@width>\linewidth\linewidth\else\Gin@nat@width\fi}
\def\maxheight{\ifdim\Gin@nat@height>\textheight\textheight\else\Gin@nat@height\fi}
\makeatother
% Scale images if necessary, so that they will not overflow the page
% margins by default, and it is still possible to overwrite the defaults
% using explicit options in \includegraphics[width, height, ...]{}
\setkeys{Gin}{width=\maxwidth,height=\maxheight,keepaspectratio}
% Set default figure placement to htbp
\makeatletter
\def\fps@figure{htbp}
\makeatother
\setlength{\emergencystretch}{3em} % prevent overfull lines
\providecommand{\tightlist}{%
  \setlength{\itemsep}{0pt}\setlength{\parskip}{0pt}}
\setcounter{secnumdepth}{-\maxdimen} % remove section numbering
\ifLuaTeX
  \usepackage{selnolig}  % disable illegal ligatures
\fi

\begin{document}
\maketitle

\documentclass[11pt]{article}
%\usepackage[portuguese]{babel}
\usepackage{natbib}
\usepackage{url}
%\usepackage[utf8x]{inputenc}
\usepackage{amsmath}
\usepackage{graphicx}
\graphicspath{{images/}}
\usepackage{parskip}
\usepackage{fancyhdr}
\usepackage{vmargin}
\setmarginsrb{3 cm}{2.5 cm}{3 cm}{2.5 cm}{1 cm}{1.5 cm}{1 cm}{1.5 cm}

\title{Milk and LDL Cholesterol level}                              % Title
\author{Huong Tran}                             % Author
\date{\today}                                           % Date

\makeatletter
\let\thetitle\@title
\let\theauthor\@author
\let\thedate\@date
\makeatother

\pagestyle{fancy}
\fancyhf{}
\rhead{\@author}
\lhead{\@title}
\cfoot{\thepage}

e\}

\begin{document}

%%%%%%%%%%%%%%%%%%%%%%%%%%%%%%%%%%%%%%%%%%%%%%%%%%%%%%%%%%%%%%%%%%%%%%%%%%%%%%%%%%%%%%%%%

\begin{titlepage}
    \centering
    \vspace*{0.5 cm}
   % \includegraphics[scale = 0.75]{logo.png}\\[1.0 cm] % University Logo
    \textsc{\LARGE \newline\newline Oakland University }\\[0.5 cm]  % University Name
    \textsc{\Large Department of Mathematics and Statistics}\\[0.5 cm]  
    \textsc{STA5229: Stat method in Sample Survey}
    \rule{\linewidth}{0.2 mm} \\[0.4 cm]
    { \huge \bfseries \@title} \\
    
    \rule{\linewidth}{0.2 mm} \\[1.5 cm]
    
    \begin{minipage}{0.5\textwidth}
        \begin{flushleft} \large
            \emph{Professor: Dorin Drignei}\\
            \end{flushleft}
            \end{minipage}~
            \begin{minipage}{0.4\textwidth}
            
            \begin{flushright} \large
            \emph{Student: Huong Tran  } \\
        \end{flushright}
        
    \end{minipage}\\[2 cm]
    
    
    \@date
    

    
    
\end{titlepage}

%%%%%%%%%%%%%%%%%%%%%%%%%%%%%%%%%%%%%%%%%%%%%%%%%%%%%%%%%%%%%%%%%%%%%%%%%%%%%%%%%%%%%%%%%

\tableofcontents
\pagebreak

%%%%%%%%%%%%%%%%%%%%%%%%%%%%%%%%%%%%%%%%%%%%%%%%%%%%%%%%%%%%%%%%%%%%%%%%%%%%%%%%%%%%%%%%%

\section{Abstract:  }
\subsection{Objective: } Consuming whole-fat dairy products might cause the increase in low-density lipoprotein (LDL) cholesterol level, which cause heart disease and heart attack.
This project identifies the affect of several types of milk on LDL cholesterol level. The comparison in LDL cholesterol level on milk-consumers will be derived. 
\subsection{Data overviews: } The data used to produce insights of the difference between several types of milk in this report is extracted from National Health and Nutrition Examination Survey (NHANES) 2017 - 2018. The following components are used:
\begin{itemize}
    \item Questionnaire Data: from component Diet and Behaviour, we extract information about the choice of milk. The following variables are used:
     \begin{itemize}
         \item DBQ223A: You drink whole or regular milk.
         \item DBQ223B: You drink 2\% fat milk.
         \item DBQ223C: You drink 1\% fat milk.
         \item DBQ223D: You drink fat free/skim milk.
         \item DBQ223E: You drink soy milk.
         \item DBQ223U: You drink another type of milk. 
     \end{itemize}
    \item Laboratory Data: In this project, we are interested in the measure of LDL cholesterol (mg/dL) using  the standard Fredewald equation, which is stored in the variable LBDLDL. Please note that, we are ignoring the requirement for triglyceride less than 400mg/dL in this project.

Note that, the cholesterol measure is valid for 12-year-older participants while the questionnaire for Diet and Behaviour are subjected for all participant. 
Therefore, weights of sample should be taken into account carefully.
    \item Demographics: 
    Because NHANES uses complex surveys designs, sample weights should be taken into account carefully: 
\begin{enumerate}
    \item Selection of PSUs, which are counties.
    Note that if counties has the population of less than 5000, it will be combined with the adjacent counties to have the required number of population.
    \item  Selection of segments within PSUs, that constitute a block/or group of blocks containing a cluster of households.
    \item  Selection of specific households with segments.
    \item Selection of individual within household.
\end{enumerate}
   Data for cholesterol level were collected in the MEC exam, therefore variable WTMEC2YR (Full sample 2 year MEC exam weight) must be used. 
\end{itemize}

\subsection{Methods: }
To observe the effect of various types of milk on consumers, we stratify the population based on the type of milk they are drinking. 
\section{Data exploratory: }
First,  we download data from NHANES website: 
\begin{verbatim}
    milk <- read.xport("dataset/DBQ_J.xpt")
    choles <- read.xport("dataset/TRIGLY_J.xpt")
    w <- read.xport("dataset/DEMO_J.xpt")

    # save as an R data frame
    saveRDS(milk, file="dataset/DEMO_I.rds")
    saveRDS(choles, file="dataset/TRIGLY_J.rds")
    saveRDS(w, file="dataset/DEMO_J.rds")
\end{verbatim}

In NHANES data, the unique identifier for each sample person is known as the sequence number (SEQN). Using SEQN, we are able to merge the three dataframes:
\begin{verbatim}
    df.tem <- merge(w, milk, by.x = "SEQN", by.y = "SEQN")
    df.join <- merge(df.tem, choles, by.x = "SEQN", by.y = "SEQN")
    myvars <- c("SEQN","DBQ223A", "DBQ223B", "DBQ223C", "DBQ223D", 
            "DBQ223E", "DBQ223U", "LBDLDL", "WTMEC2YR")
    df <- df.join[myvars]
\end{verbatim}
Some statistics of LDL Cholesterol:
\begin{verbatim}
    summary(df$LBDLDL)
\end{verbatim}
The result is as follow:
\begin{center}
    \begin{tabular}{|c|c|c|c|c|c|c|}
\hline
   Min. &1st Qu. &  Median &    Mean &3rd Qu. &    Max. &    NA's \\
   \hline
   18.0  &  82.0  & 103.0  & 106.9   & 128.0   & 357.0   &  228  \\
   \hline
\end{tabular}
\end{center}
\section{Comparison between the types of milk: }
\section{Further question: }
\newpage
\bibliographystyle{plain}
\bibliography{biblist}
 
\end{document}

\end{document}
